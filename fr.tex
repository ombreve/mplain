% Adaptation à quelques usages de la typographie française

% Ponctuations doubles.

\def\dblpspace{\hskip.5\fontdimen2\font\relax}
\catcode`;=\active
\def;{\ifhmode\ifdim\lastskip>1sp\unskip\penalty\@M\dblpspace\fi\fi\string;}
\catcode`!=\active
\def!{\ifhmode\ifdim\lastskip>1sp\unskip\penalty\@M\dblpspace\fi\fi\string!}
\catcode`?=\active
\def?{\ifhmode\ifdim\lastskip>1sp\unskip\penalty\@M\dblpspace\fi\fi\string?}
\def\colonspace{\space}
\catcode`:=\active
\def:{\ifhmode\ifdim\lastskip>1sp\unskip\penalty\@M\colonspace\fi\fi\string:}

% Guillemets

\def\guillspace{\hskip.8\fontdimen2\font
  plus .3\fontdimen3\font minus .8\fontdimen4\font\relax}
\def\og{\leavevmode\guillemetleft\penalty\@M\guillspace\ignorespaces}
\def\fg{\ifdim\lastskip>\z@\unskip\fi\penalty\@M\guillspace\guillemetright}

\mubyte \og ^^c2^^ab\endmubyte         % «
\mubyte \fg ^^c2^^bb\endmubyte         % »
\mubyte \og ^^c2^^ab^^c2^^a0\endmubyte % « suivi de nbsp
\mubyte \fg ^^c2^^a0^^c2^^bb\endmubyte % » précédé de nbsp

% Petites capitales sans césure pour les noms de famille et autres.
\font\tencsc=ec-lmcsc10
\def\bsc#1{\hbox{\tencsc #1}}

% Majuscules droites en mode mathématique.
\begingroup
  \count0="41
  \loop
    \count1=\count0 \advance\count1 by "7000
    \global\mathcode\count0=\count1
  \ifnum\count0 < "5A \advance\count0 by \@ne \repeat
\endgroup

% <= et >=.
\let\le=\leqslant
\let\ge=\geqslant

% Césures et espacement français.
\french \frenchspacing

\endinput
