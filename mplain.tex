% This is the mplain, Mostly (Modern?) Plain pdfTeX format.
% The version number and format name is redefined at the end of this file.
% See mtex.{tex,pdf} for documentation.

% Assume ini mode and set up a minimal \catcode r\'egime.
\catcode`\{=1 \catcode`\}=2 \catcode`\#=6 \catcode`\^=7 \catcode`@=11

% Ensure we're running pdfTex in extended mode and with enc extension.
\ifx\pdftexversion\undefined
  \immediate\write16{ ! fatal error:
     pdfTeX is required to process this file.}
  \batchmode\read -1 to \@tmpa
\fi
\ifx\numexpr\undefined
  \immediate\write16{ ! fatal error:
     extended mode is required to process this file.}
  \batchmode\read -1 to \@tmpa
\fi
\ifx\xprncode\undefined
  \immediate\write16{ ! fatal error:
     the 'enc' extension of TeX is required to process this file.}
  \batchmode\read -1 to \@tmpa
\fi

% Set pdfTeX basic registers.
\pdfoutput           = 1
\pdfminorversion     = 5
\pdfobjcompresslevel = 2
\pdfcompresslevel    = 9
\pdfdecimaldigits    = 3
\pdfpkresolution     = 600
\pdfhorigin          = 1 true in
\pdfvorigin          = 1 true in

% Load the 16 plain.tex fonts in lmodern version.
\font\tenrm=rm-lmr10
\font\sevenrm=rm-lmr7
\font\fiverm=rm-lmr5
\font\teni=lmmi10
\font\seveni=lmmi7
\font\fivei=lmmi5
\font\tensy=lmsy10
\font\sevensy=lmsy7
\font\fivesy=lmsy5
\font\tenex=lmex10
\font\tenbf=rm-lmbx10
\font\sevenbf=rm-lmbx7
\font\fivebf=rm-lmbx5
\font\tentt=rm-lmtt10
\font\tensl=rm-lmro10
\font\tenit=rm-lmri10

% Load plain.tex, skipping \font directives.
\def\@tmpa{\begingroup \catcode`\%=12 \catcode13=12 \@tmpb}
\begingroup
  \lccode1=13 \lowercase{\gdef\@tmpb#1^^A{\endgroup}}
\endgroup
\let\@tmpc=\font \let\font=\@tmpa
\input plain
\catcode`@=11
\let\font=\@tmpc

% Add some useful definitions.
\long\def\@gobble#1{}
\long\def\@firstofone#1{#1}
\long\def\@firstoftwo#1#2{#1}
\long\def\@secondoftwo#1#2{#2}
\def\@car#1#2\@nil{#1}
\def\@cdr#1#2\@nil{#2}

% At this point, we have a plain pdfTeX format with lmodern fonts instead of
% CM fonts. We modify this plain format, mainly to typeset non english texts.
\let\otenrm=\tenrm \font\tenrm=ec-lmr10  % lm fonts in T1/ec encoding
\let\otenbf=\tenbf \font\tenbf=ec-lmbx10
\let\otentt=\tentt \font\tentt=ec-lmtt10
\let\otensl=\tensl \font\tensl=ec-lmro10
\let\otenit=\tenit \font\tenit=ec-lmri10
\input ec                                % T1/ec encoding for text
\input u8                                % utf8 input to T1/ec
\input hy                                % hyphenation patterns
\input ma                                % fonts and definitions for maths
\input a4                                % A4 paper format
\input fr                                % french typography

% Satinize, set format name and version, and dump.
\let\@tmpa=\undefined \let\@tmpb=\undefined \let\@tmpc=\undefined
\catcode`@=12
\rm
\def\fmtname{mplain (MostlyPlain)}
\edef\fmtversion{0.577 (plain \fmtversion)}
\dump
\endinput

